\documentclass[a4paper,12pt]{article}

\usepackage[utf8]{inputenc}
\usepackage[T1]{fontenc}
\usepackage{german}
\usepackage{amsmath}
\usepackage{amssymb}
\usepackage{epsfig}
\usepackage{graphicx}
\usepackage{color}
\usepackage{subfigure}
\usepackage{tabularx}
\usepackage{cancel}
\usepackage{enumitem}
\usepackage{units}
\usepackage[left=1.5cm,right=1.5cm,top=2cm,bottom=2cm]{geometry}


%%%%%%%%%%%%%%%%%%%%%%%%%%%%%%%%%%%%%%%%%%

\newcounter{TeilPunkte}
\newcounter{AufgabePunkte}
\newcounter{GesamtPunkte}
\setcounter{GesamtPunkte}{0}

% Benutzung: \Pkte{2} vergibt 2 Punkte
%            \Pkte[6]{2} zeigt 2-Punkte-Kasten, aber vergibt 6 Punkte
\newcommand{\Pkte}[2][-999]{
    \fbox{\textcolor{black}{\textbf{#2\,P.}}}
    \ifnum#1>-999
    \addtocounter{TeilPunkte}{#1}
    \addtocounter{AufgabePunkte}{#1}
    \addtocounter{GesamtPunkte}{#1}
    \else
    \ifnum#2>0
    \addtocounter{TeilPunkte}{#2}
    \addtocounter{AufgabePunkte}{#2}
    \addtocounter{GesamtPunkte}{#2}
    \fi
    \fi
}
\newcommand{\SumPkte}{
    \dotfill \textcolor{black}{$\mathbf{\Sigma}$: \textbf{\arabic{TeilPunkte}\,P.}}
    \setcounter{TeilPunkte}{0}
}
\newcommand{\AufgabePkte}{
    \hrulefill \textcolor{black}{\large $\mathbf{\Sigma}$ Aufgabe: \textbf{\arabic{AufgabePunkte}\,P.}}\\[2ex]
    \setcounter{TeilPunkte}{0}
    \setcounter{AufgabePunkte}{0}
}
\newcommand{\GesamtPkte}{
    \hfill \fbox{\textcolor{black}{\Large Gesamtpunktzahl Eingangstest: \textbf{\arabic{GesamtPunkte}\,P.}}}
}

\newenvironment{Loesung}{
    \setcounter{TeilPunkte}{0}
    \setcounter{AufgabePunkte}{0}
    \begin{enumerate}
}{
    \SumPkte
    \end{enumerate}
    
    \AufgabePkte
    \setcounter{AufgabePunkte}{0}
}

\newcommand{\lsgitem}{
\ifnum\value{TeilPunkte}>0
\SumPkte
\fi
\item
}

%%%%%%%%%%%%%%%%%%%%%%%%%%%%%%%%%%%%%%%%%%%%%%%%%%

\setlength{\parskip}{2ex}
\setlength{\parindent}{0ex}
\setlength{\leftmargin}{0ex}

\newcolumntype{C}[1]{>{\centering\arraybackslash}m{#1}} %Kommando zum mittigen platzieren des Textes in Zellen

\newcommand{\diff}[3][]{\frac{\mathrm{d}^{#1}#2}{\mathrm{d}{#3}^{#1}}}

\newcommand{\abb}{\\[0,5cm]}
\newcommand{\ds}{\displaystyle}

\begin{document}
\begin{center}
\textbf{\large Praktikum-ET1}

\textbf{\large Eingangstest zum Versuch Test (WS 2021/22)}

\small (0102)
\end{center}

\vspace{0.5cm}

\fbox{\parbox{\textwidth}{
\vspace{4ex}

Name: \rule{7cm}{1pt} \hspace{2ex} Gruppe: \rule{2cm}{1pt} \hfill Datum: \rule{3cm}{1pt}

\vspace{1cm}

\hfill Punkte: \rule{1.9cm}{1pt}/\rule{1.9cm}{1pt} \hspace{1cm} Note: \rule{2cm}{1pt} \hspace{1cm} Signum: \rule{2.5cm}{1pt}
}}

\vspace{0.5cm}


\begin{enumerate}[leftmargin=*]

\item
Dies ist eine Aufgabe für das Praktikum.
Der Professor meint, sie sei nicht schwer zu lösen.
Der Student findet sie aber unlösbar.
\begin{enumerate}
\item Lorem ipsum dolor sit amet, consetetur sadipscing elitr, sed diam nonumy eirmod tempor invidunt ut labore et dolore magna aliquyam erat, sed diam voluptua \Pkte{1}. At vero eos et accusam et justo duo dolores et ea rebum. \Pkte{2}
\item Stet clita kasd gubergren, no sea takimata sanctus est Lorem ipsum dolor sit amet. Lorem ipsum dolor sit amet, consetetur sadipscing elitr \Pkte{1}, sed diam nonumy eirmod tempor invidunt ut labore et dolore magna aliquyam erat, sed diam voluptua. \Pkte{2}
\item At vero eos et accusam et justo duo dolores et ea rebum \Pkte{3}. Stet clita kasd gubergren, no sea takimata sanctus est Lorem ipsum dolor sit amet. \Pkte{2}
\end{enumerate}


\item
Dies ist eine Aufgabe für das Praktikum.
Der Professor meint, sie sei nicht schwer zu lösen.
Der Student findet sie aber unlösbar.
\begin{enumerate}
\item Lorem ipsum dolor sit amet, consetetur sadipscing elitr, sed diam nonumy eirmod tempor invidunt ut labore et dolore magna aliquyam erat, sed diam voluptua \Pkte{1}. At vero eos et accusam et justo duo dolores et ea rebum. \Pkte{2}
\item Stet clita kasd gubergren, no sea takimata sanctus est Lorem ipsum dolor sit amet. Lorem ipsum dolor sit amet, consetetur sadipscing elitr \Pkte{1}, sed diam nonumy eirmod tempor invidunt ut labore et dolore magna aliquyam erat, sed diam voluptua. \Pkte{2}
\item At vero eos et accusam et justo duo dolores et ea rebum \Pkte{3}. Stet clita kasd gubergren, no sea takimata sanctus est Lorem ipsum dolor sit amet. \Pkte{2}
\end{enumerate}


\item
Dies ist eine Aufgabe für das Praktikum.
Der Professor meint, sie sei nicht schwer zu lösen.
Der Student findet sie aber unlösbar.
\begin{enumerate}
\item Lorem ipsum dolor sit amet, consetetur sadipscing elitr, sed diam nonumy eirmod tempor invidunt ut labore et dolore magna aliquyam erat, sed diam voluptua \Pkte{1}. At vero eos et accusam et justo duo dolores et ea rebum. \Pkte{2}
\item Stet clita kasd gubergren, no sea takimata sanctus est Lorem ipsum dolor sit amet. Lorem ipsum dolor sit amet, consetetur sadipscing elitr \Pkte{1}, sed diam nonumy eirmod tempor invidunt ut labore et dolore magna aliquyam erat, sed diam voluptua. \Pkte{2}
\item At vero eos et accusam et justo duo dolores et ea rebum \Pkte{3}. Stet clita kasd gubergren, no sea takimata sanctus est Lorem ipsum dolor sit amet. \Pkte{2}
\end{enumerate}


\item
Dies ist eine Aufgabe für das Praktikum.
Der Professor meint, sie sei nicht schwer zu lösen.
Der Student findet sie aber unlösbar.
\begin{enumerate}
\item Lorem ipsum dolor sit amet, consetetur sadipscing elitr, sed diam nonumy eirmod tempor invidunt ut labore et dolore magna aliquyam erat, sed diam voluptua \Pkte{1}. At vero eos et accusam et justo duo dolores et ea rebum. \Pkte{2}
\item Stet clita kasd gubergren, no sea takimata sanctus est Lorem ipsum dolor sit amet. Lorem ipsum dolor sit amet, consetetur sadipscing elitr \Pkte{1}, sed diam nonumy eirmod tempor invidunt ut labore et dolore magna aliquyam erat, sed diam voluptua. \Pkte{2}
\item At vero eos et accusam et justo duo dolores et ea rebum \Pkte{3}. Stet clita kasd gubergren, no sea takimata sanctus est Lorem ipsum dolor sit amet. \Pkte{2}
\end{enumerate}




\end{enumerate}

\GesamtPkte

\end{document}
